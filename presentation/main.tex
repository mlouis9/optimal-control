% !TeX root = main.tex


% Copyright 2014 by Emmanuel Boidot <eboidot3@gatech.edu>
\documentclass[10pt,compress]{beamer}

%%%% GT theme %%%%
% add 'gold' option for golden frame titles
% partToc (resp. sectionToc) creates table of contents at the beginning of each part (resp. section)
\usetheme[sectionToc,partToc]{GT} 
\usepackage{graphicx}
\usepackage[footnotesize]{subfigure}
\usepackage[english]{babel}
\usepackage[latin1]{inputenc}
\usepackage{hyperref}
\usepackage{siunitx}

\newcommand{\bs}[1]{\boldsymbol{#1}}
\newcommand{\D}[3]{\frac{\partial^{#1} #2}{\partial^{#1} #3}}
\newcommand{\tab}{\hspace{5mm}}
\newcommand{\tb}[1]{\textbf{#1}}

\usepackage{tcolorbox}
\usepackage{xcolor}
\usepackage{physics}
\usepackage{subcaption}
\usepackage{graphicx}
\usepackage{natbib} % Change the style as needed
\newcommand{\ex}[1]{\left\langle #1\right\rangle}
\newcommand{\pa}[1]{\left(#1\right)}

\bibliographystyle{plain}



\title{Optimal Control for Nuclear Reactors}

% - Give the names in the same order as the appear in the paper.
% - Use the \inst{?} command only if the authors have different
%   affiliation.
\author[M. Louis]% (optional, appears in the lower left part of each frame)
{Calculus of Variations Final Presentation\texorpdfstring{\\}{}Matthew Louis}

\date{\today}

\begin{document}

{ % for the title frame, use the following options
\usebackgroundtemplate{\includegraphics[width=\paperwidth]{images/logos/Georgia-Tech-Insignia-Watermark-1200x1100}}
\setbeamertemplate{headline}{}
\setlength{\headheight}{0in}
\setbeamertemplate{footline}{}

% ----------
% Titlepage
% ----------

\begin{frame}
\titlepage
\end{frame}
}
\addtocounter{framenumber}{-1}


%%%%%%%%%%%%%%%%%%%%%%%%%%%%%%%%%%%%%%%% 
% \part[Intro]{Introduction} 

%%%%%
\begin{frame}\frametitle{Basics of Nuclear Power}
    \begin{enumerate}
        \item Just boils water to generate steam
        \item Energy released via fission chain reactions in fissile Uranium 235 (in fuel rods)
        \item Self-sustaining chain reaction that's stable due to negative feedback
    \end{enumerate}
    \begin{figure}
        \includegraphics[width=0.4\textwidth]{images/core-cutaway.jpg}
    \end{figure}
\end{frame}


\begin{frame}\frametitle{Motivation}
We would like to understand how to best control the chain reaction to acheive a desired end state, e.g.
\begin{enumerate}
    \item Power uprate
    \item Reactor startup
    \item Minimize power peaking in the core
\end{enumerate}
Nominally operate at \emph{steady state}, but these state changes are necessarily \emph{dynamic}. 
Safety and performance relevant. Optimal control used historically to investigate these problems.

\begin{block}{Control Mechanisms}
In light water reactors (LWRs), there are two main control mechanisms
\begin{enumerate}
    \item Control rods: controlling \emph{rapid} changes in the chain reaction
    \begin{enumerate}
        \item Relevant for reactor dynamics
    \end{enumerate}
    \item Chemical shim: controlling \emph{slow} changes in in the chain reaction
\end{enumerate}
\end{block}

\begin{columns}

\end{columns}
\end{frame}

\begin{frame}\frametitle{Relevant Quantities in Reactor Physics}
Interested only in neutrons and their interations (fission, absorption, etc).
\begin{block}{Quantities}
    \begin{itemize}
        \item $\phi$ (the scalar flux): number of neutrons impinging on a cross sectional area per second.
        \item $\Sigma$ (the macroscopic cross section): characterizes the probabilitiy for a given interaction (measured in units of 1/L)
    \end{itemize}
\end{block}
\begin{block}{The Reaction Rate}
    With the scalar flux and the cross section, we can calculate reaction rates (e.g. of fission)
    \begin{equation}
        R_x = \Sigma_x \phi\nonumber
    \end{equation}
    units of \si{1/cm^3} (a reaction rate density), and subsequently the \emph{power}
    \begin{equation}
        P =``\int_V \Sigma_x \phi dV "\nonumber
    \end{equation}
\end{block}
\end{frame}

\begin{frame}\frametitle{Fission Chain Reactions and Controllability}
    \begin{block}{Controlling the Chain Reaction}
        The fission process is incredibly fast, and in a nuclear reactor, the time between a neutron's birth via fission and its next fission reaction is
        on the order of \si{\mu s}.
        \begin{enumerate}
            \item This would make controlling a reactor nearly impossible
            \item The response to small changes in the neutron multiplication factor result in \emph{extremely} rapid changes to power
        \end{enumerate}
    \end{block}
    \begin{block}{Prompt and Delayed Neutrons}
        During a fission event, some neutrons are released (nearly) immediately from the fissioning nucleus: \tb{prompt neutrons}, and some are emitted by subsequent decaay of
        fission products \tb{delayed neutrons}.
        \begin{flushleft}
        $\beta$: The fraction of fission neutrons that are delayed ($\approx 0.7\%$)
        \end{flushleft}
    \end{block}
\end{frame}

\begin{frame}\frametitle{Governing Equations}
Simplification of the general Boltzman equation from statistical mechanics
\begin{block}{The Linear Boltzman (Neutron Transport) Equation}
    State space $\bs{r}$, $\bs{\Omega}$, $E$, $t$ (or $\bs{r}$, $\bs{v}$, $t$) $\implies \psi(\bs{r}, \bs{\Omega}, E, t)$, the angular flux
    \begin{equation}
        \begin{split}
            \frac{1}{v}\D{}{\psi}{t} &= \hat{T}(t)\psi + \hat{F}_p(t) \int_{4\pi}\psi d\bs{\Omega} + \sum_{i=1}^6 \varepsilon_i(\bs{r}, E, t) - \hat{L}(t)\psi\\
            \D{}{\varepsilon_i}{t} &= -\lambda_i \varepsilon_i + \hat{F}_{d,i}\int_{4\pi}\psi d\bs{\Omega}\tab i=1,\cdots, 6
        \end{split}
    \end{equation}
\end{block}
\begin{block}{The Operators}
    All linear operators
    \begin{columns}
        \begin{column}{5cm}
            \begin{enumerate}
                \item $\hat{T}$: Scattering
                \item $\hat{F}_p$: Prompt fission
                \item $\hat{F}_d$: Delayed fission
            \end{enumerate}
        \end{column}
        \begin{column}{5cm}
            \begin{enumerate}
                \setcounter{enumi}{3}
                \item $\hat{L}$: Leakage
            \end{enumerate}
            Note that $\hat{F}_p + \hat{F}_d = \hat{F}_{tot}$
        \end{column}
    \end{columns}
\end{block}
\end{frame}

\begin{frame}\frametitle{The Factorization}
To derive a simplified model of the transport equation that's suitable for practical calculations, we need to \emph{factorize}.
\begin{block}{Motivation}
    The main idea is to separate the variation of the flux into a fast and slow component (due to delayed neutrons). Factorization of the form
    \begin{equation}
        \psi(\bs{r}, \bs{\Omega}, E, t)=A(t)\Psi(\bs{r}, \bs{\Omega}, E; t)\nonumber
    \end{equation}
    $A(t)$ called the \emph{amplitude} function and $\Psi(\bs{r}, \bs{\Omega}, E; t)$ the shape function.
\end{block}
This factorization is \emph{always} possible (not excluding any variables) and is not (yet) unique.
\end{frame}

\begin{frame}\frametitle{The Shape Equations}
    Substituting the factorization into the transport equation
    \begin{equation}
        \begin{split}
            &A(t)\frac{1}{v}\D{}{\Psi}{t} + \Psi \frac{1}{v}\frac{dA}{dt} \\
            &= A(t)\hat{T}(t)\Psi + A(t)\hat{F}_p(t) \int_{4\pi}\Psi d\bs{\Omega} + \sum_{i=1}^6 \varepsilon_i(\bs{r}, E, t) - A(t)\hat{L}(t)\Psi\nonumber\\
            \D{}{\varepsilon_i}{t} &= -\lambda_i \varepsilon_i + A(t)\hat{F}_{d,i}\int_{4\pi}\Psi d\bs{\Omega}\tab i=1,\cdots, 6
        \end{split}
    \end{equation}
    Called the shape equations, however, $A(t)$ is an unknown so as it stands these are not well-posed.
\end{frame}

\begin{frame}\frametitle{The Adjoint Equation}
    \begin{block}{The Reference Reactor}
        First define a reference reactor in steady state. The steady state problem is
        \begin{equation}
            \hat{L}_0(t)\psi_0=\hat{T}_0(t)\psi_0 +\frac{1}{k}\hat{F}_{tot,0}\int_{4\pi}\psi_0 d\bs{\Omega}\nonumber
        \end{equation}
        $k$ is a constant necessary for ensuring a solution exists. Physically it represents the neutron multiplication factor.
        For a critical reactor, $k=1$. Introduce the ``transport operator'' $\hat{H}_0$
        \begin{equation}
            (-\hat{L}_0 - \hat{T}_0 + \hat{F}_{tot,0})\psi_0 \equiv \hat{H}_0 \psi_0 =0\nonumber
        \end{equation}
    \end{block}
    \begin{block}{The Adjoint Equation}
        Define the inner product as the integral over the entire phase space (excluding $t$), then can define adjoint equation
        \begin{equation}
            \hat{H}_0^\dagger\psi_0^\dagger =0\nonumber
        \end{equation}
    \end{block}
\end{frame}

\begin{frame}\frametitle{Projecting Onto the Adjoint}
    $\psi_0^\dagger$ can be interpreted as an importance function.
    \begin{equation}
        \begin{split}
            &A(t)\braket{\psi_0^\dagger}{\frac{1}{v}\D{}{\Psi}{t}} + \braket{\psi_0^\dagger}{\frac{1}{v}\Psi} \frac{dA}{dt} \\
            &= A(t)\braket{\psi_0^\dagger}{\hat{T}(t)\Psi} + A(t)\braket{\psi_0^\dagger}{\hat{F}_p(t) \int_{4\pi}\Psi d\bs{\Omega}} + \cdots\\
            &+ \sum_{i=1}^6 \braket{\psi_0^\dagger}{\varepsilon_i(\bs{r}, E, t)} - A(t)\braket{\psi_0^\dagger}{\hat{L}(t)\Psi}\nonumber\\
            \braket{\psi_0^\dagger}{\D{}{\varepsilon_i}{t}} &= -\lambda_i \braket{\psi_0^\dagger}{\varepsilon_i} + A(t)\braket{\psi_0^\dagger}{\hat{F}_{d,i}\int_{4\pi}\Psi d\bs{\Omega}}\tab i=1,\cdots, 6
        \end{split}
    \end{equation}
\end{frame}

\begin{frame}\frametitle{Projecting Onto the Adjoint (Continued)}
    The adjoint ($\psi_0^\dagger$) is assumed constant in time, and so inner product and time derivatives can be interchanged.
    \begin{equation}
        \begin{split}
            &A(t)\D{}{}{t}\braket{\psi_0^\dagger}{\frac{1}{v}\Psi} + \braket{\psi_0^\dagger}{\frac{1}{v}\Psi} \frac{dA}{dt} \\
            &= A(t)\braket{\psi_0^\dagger}{\hat{T}(t)\Psi} + A(t)\braket{\psi_0^\dagger}{\hat{F}_p(t) \int_{4\pi}\Psi d\bs{\Omega}} + \cdots\\
            &+ \sum_{i=1}^6 \braket{\psi_0^\dagger}{\varepsilon_i(\bs{r}, E, t)} - A(t)\braket{\psi_0^\dagger}{\hat{L}(t)\Psi}\nonumber\\
            \D{}{}{t}\braket{\psi_0^\dagger}{\varepsilon_i} &= -\lambda_i \braket{\psi_0^\dagger}{\varepsilon_i} + A(t)\braket{\psi_0^\dagger}{\hat{F}_{d,i}\int_{4\pi}\Psi d\bs{\Omega}}\tab i=1,\cdots, 6
        \end{split}
    \end{equation}
\end{frame}

\begin{frame}\frametitle{Making the Factorization Unique}
    Since the factorization is not yet uniquely defined, we may make it uniquely defined by
    choosing the convenient condition
    \begin{equation}
        \D{}{}{t}\braket{\psi_0^\dagger}{\frac{1}{v}\Psi}=0\nonumber
    \end{equation}
    which allows us to eliminate the first term in the projected equations. If we divide each term by the quantity
    $\braket{\psi_0^\dagger}{\hat{F}_{tot,0}\Psi}$ (the importance of the fission neutrons distributed according to the shape $\Psi$).
    \begin{equation}
        \begin{split}
            \frac{\braket{\psi_0^\dagger}{\frac{1}{v}\Psi}}{\braket{\psi_0^\dagger}{\hat{F}_{tot,0}\Psi}} \frac{dA}{dt} &= A(t)\frac{\braket{\psi_0^\dagger}{(\hat{T}(t)-\hat{L}(t) + \hat{F}_p(t))\Psi}}{\braket{\psi_0^\dagger}{\hat{F}_{tot,0}\Psi}} +\sum_{i=1}^6\frac{ \braket{\psi_0^\dagger}{\varepsilon_i(\bs{r}, E, t)}}{\braket{\psi_0^\dagger}{\hat{F}_{tot,0}\Psi}} \nonumber\\
            \D{}{}{t}\frac{\braket{\psi_0^\dagger}{\varepsilon_i}}{\braket{\psi_0^\dagger}{\hat{F}_{tot,0}\Psi}} &= -\lambda_i \frac{\braket{\psi_0^\dagger}{\varepsilon_i}}{\braket{\psi_0^\dagger}{\hat{F}_{tot,0}\Psi}} + A(t)\frac{\braket{\psi_0^\dagger}{\hat{F}_{d,i}\int_{4\pi}\Psi d\bs{\Omega}}}{\braket{\psi_0^\dagger}{\hat{F}_{tot,0}\Psi}}\tab i=1,\cdots, 6
        \end{split}
    \end{equation}
\end{frame}

\begin{frame}\frametitle{The Physical Interpretations}
    \begin{itemize}
        \item \begin{flushleft}
            The effective generation time of prompt neutrons
            \begin{equation}
                \Lambda(t)=\frac{\braket{\psi_0^\dagger}{\frac{1}{v}\Psi}}{\braket{\psi_0^\dagger}{\hat{F}_{tot,0}\Psi}}\nonumber
            \end{equation}
        \end{flushleft}
        \item \begin{flushleft}
            The Reactivity (related to the multiplication factor)
            \begin{equation}
                \rho(t)=\frac{\braket{\psi_0^\dagger}{\hat{H}(t)\Psi}}{\braket{\psi_0^\dagger}{\hat{F}_{tot,0}\Psi}}\nonumber
            \end{equation}
        \end{flushleft}
        \item \begin{flushleft}
            The effective delayed neutron fraction for the $i$th group
            \begin{equation}
                \tilde{\beta}_i(t)=\frac{\braket{\psi_0^\dagger}{\hat{F}_{d,i}\Psi}}{\braket{\psi_0^\dagger}{\hat{F}_{tot,0}\Psi}}\nonumber
            \end{equation}
        \end{flushleft}
    \end{itemize}
\end{frame}

\begin{frame}\frametitle{The Physical Interpretations (Continued)}
    \begin{itemize}
        \item \begin{flushleft}
            The total effective delayed neutron fraction
            \begin{equation}
                \tilde{\beta}(t) = \sum_{i=1}^6 \tilde{\beta}_i(t)\nonumber
            \end{equation}
        \end{flushleft}
        \item \begin{flushleft}
            The total effective delayed neutron emission for the $i$th group
            \begin{equation}
                \tilde{C}_i(t)=\frac{\braket{\psi_0^\dagger}{\varepsilon_i(t)}}{\braket{\psi_0^\dagger}{\hat{F}_{tot,0}\Psi}}\nonumber
            \end{equation}
        \end{flushleft}
    \end{itemize}
    Using these parameters, we can rewrite the projected equations
    \begin{equation}
        \begin{split}
            \frac{dA(t)}{dt} &= \frac{\rho(t)- \tilde{\beta}(t)}{\Lambda(t)}A(t) + \sum_{i=1}^6\lambda_i \tilde{C}_i(t)\nonumber\\
            \D{}{}{t}\tilde{C}_i(t) &= -\lambda_i \tilde{C}_i(t) + \frac{\tilde{\beta}_i(t)}{\Lambda(t)}A(t)\tab i=1,\cdots, 6
        \end{split}
    \end{equation}
\end{frame}

\begin{frame}\frametitle{The Point Kinetic Model}
    \begin{itemize}
        \item The above system of equation has a nonlinear coupling through the shape function
        \item Very hard to solve
        \item A simplified model can be obtained by neglecting spatial dependence
    \end{itemize}
    Assuming that
    \begin{equation}
        \begin{split}
            \Psi(\bs{r}, \bs{\Omega}, E;t) \approx \Psi(\bs{r}, \bs{\Omega}, E;t=0) &= \psi_0(\bs{r}, \bs{\Omega}, E)\nonumber\\
            A(t=0)&=1
        \end{split}
    \end{equation}
    The coefficients in the shape equation become constants. Usually rescale the equations so that $A(t)\to P(t)$ i.e. the total power.
    \begin{block}{The Point Kinetic Equations}
        \begin{equation}
            \begin{split}
                \frac{dP(t)}{dt} &= \frac{\rho- \tilde{\beta}}{\Lambda}P(t) + \sum_{i=1}^6\lambda_i \tilde{C}_i(t)\nonumber\\
                \D{}{}{t}\tilde{C}_i(t) &= -\lambda_i \tilde{C}_i(t) + \frac{\tilde{\beta}_i}{\Lambda}P(t)\tab i=1,\cdots, 6
            \end{split}
        \end{equation}
    \end{block}
\end{frame}

\begin{frame}\frametitle{Feedback}
\begin{itemize}
    \item The parameters $\Lambda$, $\lambda_i$, $\tilde{\beta}$, $\tilde{\beta}_i$ determined from nuclear data
    \item Cannot be solved without $\rho$.
        \begin{itemize}
            \item Typically treated as a function of time $\rho(t)= \rho_{ex}(t) + \rho_{f}(t)$
            \item Feedback from fuel/moderator temperature on reactivity via doppler broadening, etc.
        \end{itemize}
    \item Practically, $\rho_{ex}(t)$ given, and $\rho_{ex}(t)$ calculated from a feeback model.
\end{itemize}
    \begin{block}{A Lumped Feedback Model}
        Temperature influenced by $P(t)$ via
        \begin{equation}
            \frac{d}{dt}\begin{pmatrix}
                T_f\\
                T_m
            \end{pmatrix}  + \begin{pmatrix}
                -b & b\\
                c & -c -d
            \end{pmatrix}\begin{pmatrix}
                T_f\\
                T_m
            \end{pmatrix} = \begin{pmatrix}
                a P(t)\\
                d T_{in}
            \end{pmatrix}\nonumber
        \end{equation}
        Reactivity influenced by temperature via
        \begin{equation}
            \rho_f(t) = \alpha_T^f (T_f - T_{f,0}) + \alpha_T^m (T_m - T_{m,0})\nonumber
        \end{equation}
        \emph{Nonlinear} feedback
    \end{block}
\end{frame}

\begin{frame}{An Example}
    Response of a reactor to a step reactivity insertion (e.g. by ejection of a control rod)
    % \begin{figure}
    %     \begin{subfigure}[t]{0.47\linewidth}
    %         \centering
    %         \includegraphics[width=\linewidth]{images/powerCase1.jpg}
    %     \end{subfigure}
    %     \begin{subfigure}[t]{0.47\linewidth}
    %         \centering
    %         \includegraphics[width=\linewidth]{images/reactivityCase1.jpg}
    %     \end{subfigure}
    %     \begin{subfigure}[b]{0.47\linewidth}
    %         \centering
    %         \includegraphics[width=\linewidth]{images/feedbackCase1.jpg}
    %     \end{subfigure}
    %     \begin{subfigure}[b]{0.47\linewidth}
    %         \centering
    %         \includegraphics[width=\linewidth]{images/tempsCase1.jpg}
    %     \end{subfigure}
    % \end{figure}
    \begin{columns}
        \begin{column}{5cm}
            \includegraphics[width=0.9\linewidth]{images/powerCase1.jpg}
            \includegraphics[width=0.9\linewidth]{images/feedbackCase1.jpg}
        \end{column}
        \begin{column}{5cm}
            \includegraphics[width=0.9\linewidth]{images/reactivityCase1.jpg}
            \includegraphics[width=0.9\linewidth]{images/tempsCase1.jpg}
        \end{column}
    \end{columns}
\end{frame}

\begin{frame}\frametitle{Formulating Control Problems}
    \begin{itemize}
        \item Our control parameter is the external reactivity $\rho_{ex}(t)$ (from the control rods)
        \item Want to acheive some change of state
        \item E.g. a power uprate $P_0\to P_1$
    \end{itemize}
    A example problem might be performing a power uprate in \emph{minimal} time. The optimal control problem is finding
    the trajectory $\rho_{ex}(t)$ that \emph{minimizes} the uprate time $T$.
    \begin{itemize}
        \item Not simply a matter of inserting the largest possible reactivity then instantly withdrawing it to acheive the desired power
            \begin{itemize}
                \item This delayed neutrons will lead to further reactivity changes that must be accounted for before the final state can
                be reached
            \end{itemize}
        \item Operational limits on reactivity insertion, and power overshoot
            \begin{itemize}
                \item Positive reactivity insertion less than $\beta$ (sub-prompt criticality)
                \item Power overshoot must be less than $\alpha P_1$, where $\alpha= 1.5$ (for example).
            \end{itemize}
    \end{itemize}
\end{frame}

\begin{frame}\frametitle{The State Space}
    The set of state variables can be written in vector form
    \begin{equation}
        \begin{split}
            \bs{X}\equiv \begin{pmatrix}
                P\\
                C_1\\
                \vdots\\
                C_6\\
                T_f\\
                T_m
            \end{pmatrix}\nonumber
        \end{split}
    \end{equation}
    where $\bs{X}$ satisfies the following veector differential equation
    \begin{equation}
        \begin{split}
            \frac{d}{dt}\bs{X}&= \begin{pmatrix}
                \frac{\rho(t) - \beta}{\Lambda} & \lambda_1 & \cdots & \lambda_6 & 0 & 0\\
                \frac{\beta_1}{\Lambda} & -\lambda_1 & \cdots & 0 & 0 & 0\\
                \vdots & \vdots & \ddots & \vdots & \vdots & \vdots\\
                \frac{\beta_6}{\Lambda} & 0 & \cdots & -\lambda_g & 0 & 0\\
                a & 0 & \cdots & 0 & - b & b\\
                 0 & 0 & \cdots & 0 & c & -c -d
            \end{pmatrix}\bs{X} + \begin{pmatrix}
                0\\
                \vdots\\
                d T_{in}
            \end{pmatrix}\nonumber
        \end{split}
    \end{equation}
\end{frame}

\begin{frame}\frametitle{The State Space (Continued)}
    \begin{block}{Caveats}
        \begin{itemize}
            \item \emph{Nonelinear} first order problem, because of the product $\rho(t)P(t)$ (since $\rho(t)$ depends on $T_f$, $T_m$).
            \item Not really fair to write as a matrix equation.
            \item If we were to write as $\bs{X}'= \bs{f}(t, \bs{X}(t), \rho_{ext}(t))$, we see that $\bs{f}$ depends on the control $\rho_{ext}(t)$.
                \begin{itemize}
                    \item This makes solving the costate equation \emph{very} difficult
                \end{itemize}
        \end{itemize}
    \end{block}
    \begin{block}{Operational Limits on the State Space}
        \begin{enumerate}
            \item $0 \leq P(t)\leq \alpha P_1$
            \item $T_f,T_m>0$
            \item $C_i >0$
            \item Might want to impose some temperature constraints, but similar to power constraint
        \end{enumerate}
    \end{block}
\end{frame}

\begin{frame}\frametitle{Formulating the Optimal Control Problem}
    Assuming the conditions on $\bs{f}$ hold, we may apply the Pontryagin minimum principle to express the optimal control $\overline{\rho}_{ex}$ that
    minimizes the uprate time $T$ as that for which
    \begin{equation}
        \begin{split}
            \bs{\lambda}'(t) &= \pa{\partial_{\bs{X}}\bs{f}}^*\bs{\lambda}(t)\nonumber\\
            \overline{\rho}_{ext}(t) &= \text{argmin}_{\rho_{ext}\in U}\ex{\bs{\lambda}(t), \bs{f}(t, \overline{\bs{X}}, \rho_{{ext}(t)})}\nonumber
        \end{split}
    \end{equation}
    \begin{columns}
        \begin{column}{5cm}
            Where $U$ is the set of all piecewise continuous functions with maximum less than $\beta$. Let $d(t) = P(t)-P_1$, then
            \begin{equation}
                \bs{\nabla}d = \begin{pmatrix}
                    \D{}{d}{P}\\
                    \D{}{d}{C_1}\\
                    \vdots\\
                    \D{}{d}{T_m}
                \end{pmatrix}=\begin{pmatrix}
                    1\\
                    0\\
                    \vdots\\
                    0
                \end{pmatrix}\nonumber
            \end{equation}
        \end{column}
        \begin{column}{5cm}
            so $C=\{\bs{x}\in \mathbb{R}^9| x_1 < P_1\}$, we see that $d=0$ on $\partial C$ and $\bs{\nabla}d\neq 0$ on $\partial C$, so
            we have the following terminal condition on the costate
            \begin{equation}
                \bs{\lambda}(T) = \begin{pmatrix}
                    1\\
                    0\\
                    \vdots\\
                    0
                \end{pmatrix}\nonumber
            \end{equation}
        \end{column}
    \end{columns}
\end{frame}

\begin{frame}\frametitle{Numerical Alternatives}
    \begin{itemize}
        \item Due to the nonlinear term in $\bs{f}$, and the dependence on $\rho_{ext}$, the costate equation cannot be solved
        separately, then used to compute the optimal control by minimizing.
        \item An alternative is to use a numerical constrained optimization algorithm to minimize the uprate time $T$ by varying
            a number of points in a discretized representation of $\rho_{ext}(t)$
        \item Didn't have time to fully work out the details of this
        \item An example (non-optimal) trajectory
    \end{itemize}
    \begin{columns}
        \begin{column}{5cm}
            \centering
            \includegraphics[width=0.9\linewidth]{images/power.png}
        \end{column}
        \begin{column}{5cm}
            \centering
            \includegraphics[width=0.9\linewidth]{images/ramp.png}
        \end{column}
    \end{columns}
\end{frame}

\begin{frame}\frametitle{References}
    \nocite{*}
    \bibliography{bibliography.bib}
\end{frame}

\end{document}
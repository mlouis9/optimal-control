% !TeX root = main.tex


% Copyright 2014 by Emmanuel Boidot <eboidot3@gatech.edu>
\documentclass[10pt,compress]{beamer}

%%%% GT theme %%%%
% add 'gold' option for golden frame titles
% partToc (resp. sectionToc) creates table of contents at the beginning of each part (resp. section)
\usetheme[sectionToc,partToc]{GT} 
\usepackage{graphicx}
\usepackage[footnotesize]{subfigure}
\usepackage[english]{babel}
\usepackage[latin1]{inputenc}
\usepackage{hyperref}
\usepackage{siunitx}

\newcommand{\bs}[1]{\boldsymbol{#1}}
\newcommand{\D}[3]{\frac{\partial^{#1} #2}{\partial^{#1} #3}}
\newcommand{\tab}{\hspace{5mm}}
\newcommand{\tb}[1]{\textbf{#1}}

\usepackage{tcolorbox}
\usepackage{xcolor}



\title[Short title] % (optional, use only with long paper titles, appears in the lower right part of each frame)
{Optimal Control for Nuclear Reactors}

% - Give the names in the same order as the appear in the paper.
% - Use the \inst{?} command only if the authors have different
%   affiliation.
\author[M. Louis]% (optional, appears in the lower left part of each frame)
{Calculus of Variations Final Presentation\texorpdfstring{\\}{}Matthew Louis}

\date{\today}


\begin{document}

{ % for the title frame, use the following options
\usebackgroundtemplate{\includegraphics[width=\paperwidth]{images/logos/Georgia-Tech-Insignia-Watermark-1200x1100}}
\setbeamertemplate{headline}{}
\setlength{\headheight}{0in}
\setbeamertemplate{footline}{}

% ----------
% Titlepage
% ----------

\begin{frame}
\titlepage
\end{frame}
}
\addtocounter{framenumber}{-1}

{
\setbeamertemplate{headline}{}
\begin{frame}\frametitle{Outline}
  \begin{columns}
  	\begin{column}{.7\textwidth}
  		\begin{tocblock}{}
  		\vspace*{1em}\tableofcontents[onlyparts]
  		\end{tocblock}		
  	\end{column}
  \end{columns}
\end{frame} 
}


%%%%%%%%%%%%%%%%%%%%%%%%%%%%%%%%%%%%%%%% 
\part[Intro]{Introduction} 

\section{Section 1}
\subsection{Subsection 1.1}
%%%%%
\begin{frame}\frametitle{Intro - Frame 1} 
\begin{block}{Basic Block Title}
\begin{itemize}
 \item This is the basic block
\item use \textbackslash begin\{block\}...\textbackslash end\{block\}
\end{itemize}
\end{block}
\pause
\begin{columns}
\begin{column}{5cm}
\begin{exampleblock}{Example Block 1}
\begin{itemize}
 \item This is an example block
\item use \textbackslash begin\{exampleblock\}\\ ...\textbackslash end\{exampleblock\}
\end{itemize}
\end{exampleblock}
\pause
\end{column}
\begin{column}{5cm}
\begin{exampleblock}{Example Block 2}
.\\
.\\
.\\
.\\
\end{exampleblock}
\end{column}
\end{columns}
\end{frame}

\subsection{Test}
\begin{frame}\frametitle{Test frame one}
\begin{alertblock}{Alerted Title}
\begin{itemize}
 \item This is an alert block
 \item Use it for questions
\item use \textbackslash begin\{exampleblock\}\\ ...\textbackslash end\{exampleblock\}
\end{itemize}
\end{alertblock}
\end{frame}

\section{Section 2}
\subsection{Subsection 2.1}
\begin{frame}\frametitle{Test frame - Subsection I.2.1}test\end{frame}
\subsection{Subsection 2.2}
\begin{frame}\frametitle{Test frame - Subsection I.2.2}test\end{frame}

\section{Section 3}
\subsection{Subsection 3.1}
\begin{frame}\frametitle{Test frame - Subsection I.3.1}test\end{frame}
\subsection{Subsection 3.2}
\begin{frame}\frametitle{Test frame - Subsection I.3.2}test\end{frame}
\subsection{Subsection 3.3}
\begin{frame}\frametitle{Test frame - Subsection I.3.3}test\end{frame}

\part{Part two}

\section[Section 1]{Test Section One with a really long title}
\begin{frame}\frametitle{Test frame two}test\end{frame}
\section{Test Section Two}
\begin{frame}\frametitle{Test frame two}test\end{frame}
\section{Test Section Three}
\begin{frame}\frametitle{Test frame three}test\end{frame}
\section{Test Section Four}
\begin{frame}\frametitle{Test frame four}test\end{frame}

\bibliographystyle{plain}
\begin{frame}[allowframebreaks]\frametitle{References}
\bibliography{../proposal_references}
\end{frame}

\end{document}